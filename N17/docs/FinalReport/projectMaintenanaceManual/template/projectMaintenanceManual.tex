\documentclass{project}
\usepackage[pdfauthor={C. P. Marriott},pdftitle={Software Engineering Group Project, Project Plan},pdftex]{hyperref}
\usepackage[pdftex]{graphicx}
\usepackage{pdfpages}
\hypersetup{colorlinks=false,pdfborder={0 0 0}}
\begin{document}
\title{Software Development Life cycle}
\subtitle{Maintenance Manual}
\author{Tom Reed, Matt Whitmore, Dave Clark, Silhab Csoma, Mike Steel, Chris 'Tux' Lloyd, Aleksandra Badyda, Samuel Jackson, Chris Marriott}
\shorttitle{Test Report}
\version{1}
\status{draft}
\date{2013-01-30}
\configref{SE.17.DS.01}
\maketitle
\tableofcontents
\newpage

\section{Maintenance Manual}
Program maintainers pick up the maintenance documentation because they have a specific question in mind.
The goal of your program maintenance documentation should be to answer all of the likely questions, or at least
to show the maintainer which part of the program source is likely to provide the answer. Examples of
maintainers� questions are:

\begin{itemize}
	\item This program crashes with a particular combination of inputs. Where is the bug likely to be, and how
do I rebuild and test the system once I have fixed it?
	\item How can I extend this �sort� program to add an option which will delete any repeated lines found while
sorting? Indeed, is its structure such as to permit such a modification without altering its design
completely?
	\item Can I speed up this slow program without major change? For example, I have found a new and
wonderful sorting procedure. Can I replace the slow sort used in this genealogical data management
program?
\end{itemize}

Most programs do in fact contain bugs. Normal commercial practice, especially with minor bugs, is to
document them and correct them at a convenient time (perhaps 3-12 months later), rather than rushing to fix
them and releasing a new version immediately. Making immediate fixes is costly in distribution and
reinstallation of new versions and above all leads to problems when the �fixes� have unforeseen side effects.
This lead-time on fixing bugs means that documenting any likely changes and how to make them is vital,
because the changes may be carried out by someone else.
A checklist for the structure of a maintenance manual is:

\subsection{Program Discription}
This will give a brief description of what the program does and how it does it, e.g. for
a sorting program you might say It sorts a list of English words into alphabetical order using the bubble
sort method.

\subsection{Program Structure}
This should describe the design of the program. Design diagrams and pseudo-code are
both useful methods of doing this. One of the most useful diagrams for a maintainer is one that shows
which routines in the program call which other routines (a flow of control diagram). A list of program
modules and their purpose should also be given. There should also be a list of methods. This specifies the
name, parameters and their types, the type returned by a method, and a brief description of what each
method does. Often a couple of lines on each module will be enough. Where this information is
contained in your design specification, it can be left out of the maintenance document, and a reference
made to the appropriate sections of the design specification.

\subsection{Algorithms}
Here you describe in detail the significant algorithms used in the program or, in the case of
well known methods, you may give references. Again, if this information is contained in the design
specification, it can just be referenced here.

\subsection{The main data areas}
This specifies the data structures, including arrays, objects etc. where important
information is stored for a substantial part of the main program. For example, in a program that adds a
student�s marks together and calculates a grade, there might be data structures used to store a student�s
project and examination marks for each course. Again, if this information is contained in the design
specification, it can just be referenced here.

\subsection{Files}
It may be that the program accesses certain fixed files or needs files of a certain type to be available.
Give such information here. For example, The program creates the file XYZ.test as workspace and later
deletes it. If such a file exists already then its contents will be lost. Another example is The program
assumes that the current directory contains a file of integers at three per line, separated by spaces.

\subsection{Interfaces}
Many programs control or read devices such as measuring instruments. Usually there will be
certain protocols to be observed, requirements that a terminal is set up in a particular way, etc. For
example, The terminal should be set to read and transmit at a baud rate of at most 1200. The
possibilities here are endless, but each application is likely to have a few simple rules that must be
observed, and such information should be given in this section.

\subsection{Suggestions for improvements}
Most programs are a compromise between what one would like to do and
what one has time to do. Where desirable improvements have had to be omitted because of constraints
such as time or the available hardware or software, it is worth suggesting them for the benefit of future
programmers tackling the same problem. Where different ways of solving some of the programming
problems have been considered, then it can be useful to others who need to work on the same program to
have this information. It may be that improvements in hardware or software mean that methods rejected
now can be used when the program is revised.

\subsection{Things to watch for when making changes}
It is desirable to avoid a programming style which means that
changes can have knock-on effects which affect other parts of the program. Sometimes this is
unavoidable. The programmer should be very careful to list any known effects of this nature.

\subsection{Physical limitations of the program}
Obviously a computer installation is a finite environment. It can only
have so much memory, so much disk space, and only allow each user so much processor time. Some
programs will come against these constraints. It is particularly important to list the requirements, where
known, because not all environments impose the same constraints - a program might run without
restriction in one environment, require special options to be chosen in another, and be completely beyond
the capabilities of a third. There is also the question of accuracy when real numbers are used. The
documentation should discuss this, if relevant, giving details of the expected accuracy of inputs, that
provided by the algorithms, and that of the output.

\subsection{Rebuilding and Testing}
Maintainers need to know what to do when rebuilding a program. Where are all
the files? What should they do to rebuild the system? How do they find out what tests to run? How do
they know whether it has passed the tests? How do they add a test when a new problem is discovered? If
documents are in a non-standard format (e.g. LaTeX rather than MS Word), then it may also be necessary
to describe how to rebuild them.
\clearpage
\addcontentsline{toc}{section}{REFERENCES}
\begin{thebibliography}{5}
\bibitem{} \emph{N/A}
\end{thebibliography}
\clearpage
\addcontentsline{toc}{section}{DOCUMENT HISTORY}
\section*{DOCUMENT HISTORY}
\begin{tabular}{|l | l | l | l | l |}
\hline
Version & CCF No. & Date & Changes made to Document & Changed by \\
\hline
1.0 & N/A & 2013-01-30 & Initial creation & dac26 \\
\hline
\end{tabular}
\label{thelastpage}
\end{document}