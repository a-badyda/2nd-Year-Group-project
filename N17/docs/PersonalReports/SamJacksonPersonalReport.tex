\documentclass{article}

\setlength{\parskip}{10pt plus 1pt minus 1pt
\setlength{\parindent}{0cm}}

\begin{document}
\section*{Personal Reflective Report}
\subsection*{Samuel Jackson}
My role in the group project was to be one of the main programmers for the front end of the site and to aid with the general design of the system. While the finished system which we produced managed to meet most of the requirements outlined in the brief, I feel that there were a lot of things that hindered the progress and development of the project more than there could of been.

On the whole, I felt that the majority of the early part of the project went reasonably well. Our team managed to schedule and attend regular meetings and could set targets and activities to each of the team members and generally kept to the deadlines set by the project manager. I also feel that people were adequately assigned to their roles according to their programming/design experience and individual preferences.

However, I feel one of the major flaws with the project was the failure to fully analyse the implications of the design we produced. This lead to the design team (myself included) designing a system that could not work in the way intended. This was largely because of concurrency issues when multiple users are logged into the system.

Subsequently, much of the back end of the system needed redesigning on the fly during the early stages of implementation and testing week. This later caused knock on delays with implementing some functional elements of the system (e.g. ageing and server-to-server communication) as well as delaying the start of the testing phase.

This coupled with the added hindrance of frequent university network outages meant that we had to make strategic practical decisions in order to get as many of the functional requirements up and running by the Friday deadline. This included actively deciding to drop some of the more subtle functional requirements in order to concentrate on the more important elements of the system (i.e server-to-server communication). If this were a real software development project, I believe that the user would either of been force to accept a lower quality of product or grant the team an extension of another couple of days.

I would like to outline at this point that if it were not for the network outages I have full confidence in that fact that our team would have been able to get at least one of the missed requirements serviceable by the deadline and quite possibly both of them.

Turning to the more positive elements of the project, the logical data structures and algorithms (such as those used for battling and breeding) which were developed in isolation from the rest of the system and integrated during coding week work straight out of the box. When it came to there integration, all that was needed was for Monster objects to be passed in and the pre-written code did the rest. This meant that more development time could be devoted to other issues with the system.

A key factor in the success of the this project was largely down to the rigour and efficiency of the testing team. Once we had the system to a working but highly buggy stage, the testing team was set loose on the project and ran through all the tests in our test plan and later also using the user acceptance tests emailed out by Bernie, as well as any other creative tests they could think of. I feel that a large part of any successes we made is a down to them.

As we were on such a tight schedule, it was imperative that feedback from the testers reached the programmers as quickly and efficiently as possible. To do this, whenever a bug was found with the system, it was written on paper under the name of the developer responsible for the code that caused it. Each developer then worked through each of the bugs while the testing was still running and attempted to fix each issue. The project would then be redeployed and testing would begin again from scratch. Any new or continuing issues would then be identified and reported. 

This cycle continued until all implemented functional requirements passed. This system worked well because it allowed all members (QA's and developers) to be active on the project at the same time, meaning it became very efficient to weed out bugs.

Another technique that was applied to the project that I felt worked quite well was the use of a pair programming technique between myself and the lead server side developer. Because of the issues already outlined in this document, we were often forced to work long hours implementing and integrating the client side and server side of the system. 

By working closely together, carefully discussing and occasionally editing each other's code, it allowed use to reduce difficulties getting the two components of the system to talk to each other. It also allowed us to brain storm effective solutions to problems encountered along the way and let us double check what we had each written. This was especially useful when writing SQL queries and JSON strings while sleep deprived where spelling and formatting are paramount.

In conclusion, I feel that the implementation of this project could have gone smoother and some aspects could have been better outlined/developed/analysed, but in general I am happy with the outcome. Furthermore, I would like to outline once again how despite the setbacks and design obstacles our team encountered with the project, I feel that all team members pulled there weight and that producing a system to the level delivered would have been impossible without the effort exhibited by every member group 17.

\end{document}
